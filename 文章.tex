\documentclass{article}
\usepackage{ctex}

\usepackage{geometry}
\newgeometry{
    top=30mm, bottom=25mm, left=30mm, right=20mm,
    headsep=5mm
}
%页边距

\usepackage{titlesec} %自定义多级标题格式的宏包
\titleformat{\section}[block]{\LARGE\bfseries}{\arabic{section}}{1em}{}[]
\titleformat{\subsection}[block]{\Large\itshape\mdseries}{\arabic{section}.\arabic{subsection}}{1em}{}[]
\titleformat{\subsubsection}[block]{\normalsize\bfseries}{\arabic{subsection}-\alph{subsubsection}}{1em}{}[]
\titleformat{\paragraph}[block]{\small\bfseries}{[\arabic{paragraph}]}{1em}{}[]

\iffalse
\titleformat{command}[shape]%定义标题类型和标题样式,字体
{format}%定义标题格式:字号(大小),加粗,斜体
{label}%定义标题的标签,即标题的标号等
{sep}%定义标题和标号之间的水平距离
{before-code}%定义标题前的内容
[after-code]%定义标题后的内容
\fi

\usepackage{graphicx} %插入图片的宏包
\usepackage{float} %设置图片浮动位置的宏包
\usepackage{subfigure} %插入多图时用子图显示的宏包


\begin{document}
\begin{titlepage}
    % 第二个()里的参数表示左移35pt,下移55pt
    \begin{picture}(0,0)(35,55)
        \includegraphics[width=90pt]{figure/buaamark.eps}
    \end{picture}
    \hfill
    
    \vskip 95bp
    \begin{center}
        \includegraphics[width=360bp]{figure/buaaname.eps}
        \vskip 45bp
        \centerline{\zihao{-0}\heiti 时间序列分析课程论文}
        ~~\\
        \vspace*{\stretch{4}}
        \begin{minipage}[h]{.8\textwidth}
            \centering{\heiti\zihao{2}时间序列分析课程论文}
        \end{minipage}
        \vskip 20bp
        \begin{minipage}[h]{.75\textwidth}
            \centering{\heiti\zihao{3}时间序列分析课程论文}
        \end{minipage}
        \vspace*{\stretch{3}}
        ~~\\
        {
            \zihao{-3}\heiti            
            \begin{tabular}{cc}
                学~~~院~~~名~~~称~~&\underline{~~~~学院~~~~}\\[.4ex]
                专~~~业~~~名~~~称~~&\underline{~~~~专业~~~~}\\[.4ex]
                学~~~生~~~姓~~~名~~&\underline{~~~李俊然~~~}\\[.4ex]
                指~~~导~~~教~~~师~~&\underline{~~~123~~~~~}\\
            \end{tabular}
            
        }
        \vskip 60bp
        \centerline{\heiti\zihao{-3}2021~~年~~5~~月~~23~~日}
    \end{center}
\end{titlepage}

\pagenumbering{gobble}
\newpage
\pagenumbering{arabic}
\section{绪论}
\subsection{问题描述}
在现代,对于气象数据的预测对于某些商业活动非常重要
\subsection{数据来源}
数据来自于Kaggle,源自Weather Undergroud API。数据包含了2013年1月1日到2017年4月24日德里(Delhi)的每日平均气温、
湿度、风速、气压数据,本模型利用了温度进行时间序列分析。
\section{数据描述}
这是第二章
\section{模型构建}
21
\section{模型预测检验}
21
\end{document}