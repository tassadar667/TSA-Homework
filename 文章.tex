\documentclass{article}
\usepackage{ctex}

\usepackage{geometry}
\geometry{a4paper,left=2cm,right=2cm,top=1cm,bottom=1cm}
%页边距

\usepackage{titlesec} %自定义多级标题格式的宏包
\titleformat{\section}[block]{\LARGE\bfseries}{\arabic{section}}{1em}{}[]
\titleformat{\subsection}[block]{\Large\itshape\mdseries}{\arabic{section}.\arabic{subsection}}{1em}{}[]
\titleformat{\subsubsection}[block]{\normalsize\bfseries}{\arabic{subsection}-\alph{subsubsection}}{1em}{}[]
\titleformat{\paragraph}[block]{\small\bfseries}{[\arabic{paragraph}]}{1em}{}[]

\iffalse
\titleformat{command}[shape]%定义标题类型和标题样式,字体
{format}%定义标题格式:字号(大小),加粗,斜体
{label}%定义标题的标签,即标题的标号等
{sep}%定义标题和标号之间的水平距离
{before-code}%定义标题前的内容
[after-code]%定义标题后的内容
\fi


\title{时间序列分析课程论文}
\date{2021年5月23日}
\author{18377198 李俊然}
 
\begin{document}
\pagenumbering{gobble}
\maketitle
\newpage
\pagenumbering{arabic}
\section{绪论}
\subsection{问题描述}
在现代,对于气象数据的预测对于某些商业活动非常重要
\subsection{数据来源}
数据来自于Kaggle,源自Weather Undergroud API。数据包含了2013年1月1日到2017年4月24日德里(Delhi)的每日平均气温、
湿度、风速、气压数据,本模型利用了温度进行时间序列分析。
\section{数据描述}
这是第二章
\section{模型构建}
21
\section{模型预测检验}
21
\end{document}